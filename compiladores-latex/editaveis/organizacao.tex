\chapter[Organização]{Organização}
\section{Estrutura Interna}
A equipe é composta por 4 (quatro) desenvolvedores, Carolina Ramalho, David Carlos, Lucas Kanashiro e Thiago Ribeiro, sendo a primeira desses a Coach da equipe.

\section{Metodologia}
A equipe optou por adotar a metodologia ágil para o projeto, com o intuito de que a cada sprint de 2 semanas possamos sair com pequenas partes do compilador.

\section{Ferramentas}
\subsection{Comunicação}
Para a comunicação a equipe optou por utilizar as seguintes ferramentas: IRCCloud, e-mail e Telegram.
\subsection{Repositório}
A equipe ciou uma organização no Github denominada UnBsafeC, na qual exstem dois repositórios, um para desenvolvimento e outro para documentação.
Desenvolvimento: https://github.com/UnBsafeC/safec.git
Documentação:https://github.com/UnBsafeC/safe-documentation.git
\subsection{Kanban}
O grupo utilizará o kanban online Waffle, que pode ser encontrado no endereço https:://www.waffle.io/safec
\subsection{Ambiente}
A configuração do ambiente de desenvolvimento será feita com uma máquina virtual utilizando Vagrant e VirtualBox, caso necessário. Será definido uma instalação padrão em uma máquina virtual para qualquer pessoa poder utilizar o ambiente de desenvolvimento do nosso compilador sem maiores problemas com sistema operacional.
\subsection{Teste}
O grupo utilizará a ferramenta CPPUnit para realizar testes unitários do código c/c++ gerado. Com isso pretendemos atingir uma cobertura aceitável de testes de no mínimo 90\% das funções em c/c++ desenvolvidas pela equipe.

\section{Gerência de Requisitos}
Os requisitos serão divididos em épicas, features e histórias[1], sendo épicas o nível mais alto e abstrato e as histórias o nível mais baixo e detalhado.

Primeiramente os requisitos estão mapeados em:
\begin{itemize}
\item Épica: Analisar o código fonte em C e identificar possíveis vulnerabilidades
\item Features: Cada feature será um cenário de vulnerabilidade.
\item  Histórias de Usuário: Conterá as histórias feitas pela equipe, no qual serão dividas em cada dupla e cada história contém um conjunto de tarefas. As atividades inicias estão listadas na sessão 3.5 .
\end{itemize}

\section{Atividades a serem desenvolvidas}
Desenvolvimento:	
\begin{itemize}
\item Definição de escopo ;
\item Estudo das vulnerabilidades ;
\item Descrever o processo para identificar os cenários de vulnerabilidades ;
\item Priorização dos cenários ;
\item Criar as regras léxicas e sintáticas ;
\item Adicionar as regras léxicas e sintáticas;
\item Testar as regras adicionadas ;
\item Gerar código comentado ;
\item Testar saídas do compilador;
\item Estudo na práticas das ferramentas Bison e Flex;
\item Escrever diagrama de classes.
\end{itemize}

Entrega de release:
\begin{itemize}
\item Confeccionar vídeo;
\item Organização do time;
\item Revisar documentação de entrega.
\end{itemize} 