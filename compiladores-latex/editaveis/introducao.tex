\chapter[Introdução]{Introdução}

\section{Problemática}
No mundo atual, onde praticamente todos estão interligados através da Internet,
devemos nos preocupar cada vez mais com a segurança dos softwares utilizados. Uma
simples falha de segurança em um software pode causar desde danos pessoais, como perca
de arquivos pessoais, até mesmo perca de capital por parte de uma empresa multinacional. Sabendo que vulnerabilidade é um conjunto de condições que podem levar a violação de
uma política de segurança explícita ou implícita [5], necessita-se identificar possíveis ameaças de vulnerabilidades de código fonte a fim de contorna-las ainda dentro do ciclo de desenvolvimento.
Como foi dito por Chess e West [4], erros são inevitáveis, mas deve existir uma forma de controla-los ainda dentro do ciclo de desenvolvimento, podendo reduzir possíveis custos futuros de manutenção corretiva do software.

\section{Justificativa}
Além dos fins acadêmicos, a construção deste compilador servirá de aprendizado de vários aspectos teóricos da disciplina de compiladores, vendo na prática da construção até o funcionamento de um compilador. Além disso a ferramenta poderá auxiliar na busca de possíveis ameaças de vulnerabilidade nos códigos fonte, podendo auxiliar programadores iniciantes levando a construção de um código mais seguro ou até a busca de vulnerabilidades em grandes aplicações.


\section{Objetivo Geral}
Este projeto tem como objetivo a construção de um compilador que seja capaz de analisar as seguintes ameaças de vulnerabilidades[2] em código escrito na linguagem C:
\begin{itemize}
\item Ameaça de Vulnerabilidade 1 - Utilização de variáveis não inicializadas.
\item Ameaça de Vulnerabilidade 2 - Referência a ponteiros nulos.
\item Ameaça de Vulnerabilidade 3 - Referência a posições de memoria já desalocadas, previamente.
\item Ameaça de Vulnerabilidade 4 - Desalocar uma região de memória já desalocada (free free).
\item Ameaça de Vulnerabilidade 5 - Não desalocação de memória.
\item Ameaça de Vulnerabilidade 6 - Divisão por 0
\begin{enumerate}
\item Simples: Divisão direta pelo valor 0 (de maneira explícito).
\item Complexa: Divisão por variáveis que são iguais a 0
\end{enumerate}
\item Ameaça de Vulnerabilidade 7 - Acesso a posições inválidas de uma array estática.
\end{itemize}

